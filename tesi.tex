%CLASSE DOCUMENTO - LINGUA E DIMENSIONE FONT
\documentclass[12pt,twoside]{toptesi}

%%%%%%%%%%%%%%%%%%%%%%%%%%%%%%%%%%%%%%%%%%%%%%%%%%%%%%%%%%%%%%%

% INCLUSIONE PACCHETTI
\usepackage[utf8]{inputenc} %utf8
\usepackage[italian]{babel}
\usepackage[T1]{fontenc}
\usepackage{blindtext}
\usepackage{graphicx,wrapfig}
\usepackage{booktabs}
\usepackage{lmodern}
\usepackage{varioref}
\usepackage{url}
\usepackage{array}
\usepackage{paralist}{\obeyspaces\global\let =\space}
\usepackage{verbatim} 
\usepackage{subfig}
\usepackage{tabularx}
\usepackage{amsmath}
\usepackage{amsfonts}
\usepackage{float}
\usepackage{amssymb}
\usepackage{multicol}
\usepackage{multirow}
\usepackage{listings}
\usepackage[pass]{geometry}
\usepackage[figuresright]{rotating}
\usepackage{algorithm}
\usepackage{algorithmic}
\usepackage{amsmath}
\usepackage[babel]{csquotes}
\usepackage{hyperref}
\usepackage[backend=bibtex]{biblatex}
\usepackage{pdfpages} 
\usepackage{pythonhighlight}
\usepackage{pdflscape}
\usepackage{longtable}
%%%%%%%%%%%%%%%%%%%%%%%%%%%%%%%%%%%%%%%%%%%%%%%%%%%%%%%%%%%%%%%

% CONFIGURAZIONE LINK E RIFERIMENTI
\hypersetup{%
    pdfpagemode={UseOutlines},
    bookmarksopen,
    pdfstartview={FitH},
    colorlinks,
    linkcolor={black}, %COLORE DEI RIFERIMENTI AL TESTO
    citecolor={black}, %COLORE DEI RIFERIMENTI ALLE CITAZIONI
    urlcolor={black} %COLORI DEGLI URL
}


\newenvironment{tightcenter}{%
  \setlength\topsep{0pt}
  \setlength\parskip{0pt}
  \begin{center}
}{%
  \end{center}
}

%%%%%%%%%%%%%%%%%%%%%%%%%%%%%%%%%%%%%%%%%%%%%%%%%%%%%%%%%%%%%%%

% CONFIGURAZIONE LISTATI/CODICE - CANCELLARE SE NON NECESSARIO
% PYTHON - BIANCO E NERO
% \lstset{%
% 	captionpos=b,
% 	language=Python,
% 	basicstyle =\small\ttfamily,
% 	keywordstyle=\color{black}\bfseries,
% 	breaklines=true,
% 	breakatwhitespace=true,
% 	frame=lines,
% 	numbers=left,
% 	numberstyle=\footnotesize,
% }

%%%%%%%%%%%%%%%%%%%%%%%%%%%%%%%%%%%%%%%%%%%%%%%%%%%%%%%%%%%%%%%

% FRENCHSPACING VA _SEMPRE_ ABILITATO PER DOCUMENTI IN ITALIANO
\frenchspacing

%%%%%%%%%%%%%%%%%%%%%%%%%%%%%%%%%%%%%%%%%%%%%%%%%%%%%%%%%%%%%%%

%DEFINIZIONE SEZIONI IN NUMERAZIONE ROMANA
%ELENCO DEI LISTATI/CODICI
\makeatletter
\newcommand\listofcodes{%
 \iffrontmatter\else\frontmattertrue\fi
 \if@openright\cleardoublepage\else\clearpage\fi
 % change the meaning of \chapter in a group
 \begingroup\def\chapter##1{\@schapter}
 \phantomsection % for the hyperlink
 \lstlistoflistings 
 \endgroup
} 
\makeatother

%%%%%%%%%%%%%%%%%%%%%%%%%%%%%%%%%%%%%%%%%%%%%%%%%%%%%%%%%%%%%%%

% INFORMAZIONI PDF - PERSONALIZZARE
\pdfinfo{%
  /Title    (Un approccio alla Timed-Release Encryption basato su Smart Contract)
  /Author   (Andrea Cattaneo)
% /Subject  (Un trattato di toccante urgenza)
% /Keywords (LaTeXi carta igienica urgenza Politecnico Presicce)
}

% %%%%%%%%%%%%%%%%%%%%%%%%%%%%%%%%%%%%%%%%%%%%%%%%%%%%%%%%%%%%%%%

% % FRONTESPIZIO - PERSONALIZZARE
% % ELIMINATE LE VOCI CHE NON VI SERVONO

% % UNIVERSITA - NOME
% \ateneo{Università degli Studi di Bergamo}

% % FACOLTA - DICITURA - CANCELLARE O DECOMMENTARE
% %\FacoltaDi{Faculty of}
% % FACOLTA - NOME
% \facolta{Ingegneria}

% % CORSO DI LAUREA - DICITURA (MANTENERE LO SPAZIO) - CANCELLARE O DECOMMENTARE
% %\CorsoDiLaureaIn{Master of Science in }
% % CORSO DI LAUREA - NOME
% \corsodilaurea{Ingegneria Informatica}

% % TIPOLOGIA TESI
% \TesiDiLaurea{Tesi di Laurea Triennale}

% % TITOLO
% \titolo{Senza carta igienica}

% % SOTTOTITOLO
% \sottotitolo{Un trattato di toccante urgenza}

% % RELATORE/I - DICITURA - CANCELLARE SE UN SOLO RELATORE
% %\AdvisorName{Relatori}
% % RELATORE - PROF. NOME E COGNOME
% \relatore{prof.\ Darth Vader}
% % RELATORE AGGIUNTIVO - PROF NOME E COGNOME
% % SE SI HA SOLO UN RELATORE ELIMINARE INSIEME AD AdvisorName
% \secondorelatore{prof.\ Neo Cortex}

% % TUTORE AZIENDALE - TITOLO NOME E COGNOME
% \tutoreaziendale{Ing. Carlino Cane}
% % TUTORE AZIENDALE - DICITURA//AZIENDA
% \NomeTutoreAziendale{Tutore Aziendale\\FeelGood Inc}

% % CANDIDATO - DICITURA (MANTENERE I DUE PUNTI) - CANCELLARE O DECOMMENTARE
% %\CandidateName{Candidate:}

% % CANDIDATO - NOME E COGNOME
% \candidato{Andrea Cattaneo}
% % CANDIDATA - ELIMINARE O SOSTITUIRE CON IL PRECEDENTE
% %\candidata{Tinasa de Tinasis} 
% % SECONDO CANDIDATO - ELIMINARE O DECOMMENTARE
% %secondocandidato{Bombo de Bombis}
% %secondacandidata{Bomba de Bombis}

% % LOGO UNIVERSITA
% \logosede{images/logo}

% % DATA - MESE ANNO
% \sedutadilaurea{Dicembre 2018}

% %%%%%%%%%%%%%%%%%%%%%%%%%%%%%%%%%%%%%%%%%%%%%%%%%%%%%%%%%%%%%%%

% LISTA DEI CAPITOLI DA INCLUDERE - PERSONALIZZARE
\includeonly{%
chapters/chap_intro,%
chapters/chap_tre,%
chapters/chap_protocollo,%
chapters/chap_poc,%
chapters/chap_analisi_attacchi,%
chapters/chap_dimensionamento,%
chapters/chap_possibili_usi,%
}

% FILE DI BIBLIOGRAFIA
\bibliography{bibliography} 


%%%%%%%%%%%%%%%%%%%%%%%%%%%%%%%%%%%%%%%%%%%%%%%%%%%%%%%%%%%%%%%

% INIZIO DOCUMENTO
\begin{document}

% \frontespizio
\includepdf{frontespizio/fronteingegnerialt.pdf}
\paginavuota

%%%%%%%%%%%%%%%%%%%%%%%%%%%%%%%%%%%%%%%%%%%%%%%%%%%%%%%%%%%%%%%

%INTERLINEA - DEFAULT 1 - NON ESAGERATE, NON SUPERATE MAI 1.3 ;)
\interlinea{1.5}

%%%%%%%%%%%%%%%%%%%%%%%%%%%%%%%%%%%%%%%%%%%%%%%%%%%%%%%%%%%%%%%

\frontmatter

% DEDICA - PERSONALIZZARE
% VSPACE - PROPORZIONE USATA PER CENTRATURA VERTICALE DEL TESTO
% FLUSHRIGHT - ALLINEAMENTO ORIZZONTALE A DESTRA
% \vspace*{\stretch{1}}
% \begin{flushright}
% 	\noindent
% 	All'amato me stesso
% \end{flushright}
% \vspace*{\stretch{6}}
% \cleardoublepage


% CITAZIONE - PERSONALIZZARE
% VSPACE - PROPORZIONE USATA PER CENTRATURA VERTICALE DEL TESTO
% FLUSHRIGHT - ALLINEAMENTO ORIZZONTALE A DESTRA
\vspace*{\stretch{1}}
\begin{flushright}
	\noindent
	If I have 300 ideas in a year and
	\\just one turns out to work I am satisfied.

	\textit{Alfred Nobel}
\end{flushright}
\vspace*{\stretch{6}}
\paginavuota


% ABSTRACT - PERSONALIZZARE
\sommario
TODO SISTEMARE

Nella prima parte di questa tesi presenteremo le reti blockchain.
Partiremo dalla storia della blockchain, presenteremo le principali tecniche per il
raggiungimento del consenso e la classificazione in blockchain
\textit{permissioned}/\textit{permissionless}. Successivamente analizzaremo alcuni dati
sul livello di adozione della tecnologia blockchain nelle organizzazioni.
Passeremo poi agli Smart Contract. Andremo poi a discutere la Timed-Release Encryption.
Successivamente presenteremo un protocollo per ottenere la Timed-Release Encryption
basato su un approccio con \textit{trusted agents}.
Il protocollo utlizza la blockchain come orologio autoritativo e alcuni
smart contract per garantire incentivi economici agli agenti coinvolti.
Inoltre ne realizzeremo una \textit{proof of concept} sullo Stellar Network.
Successivamente proporremo un modello per un analisi quantitativa della robustezza
delle implementazioni del protocollo.
Nella parte finale vedremo alcuni possibili usi della Timed-Release Encryption.
\paginavuota


%%%%%%%%%%%%%%%%%%%%%%%%%%%%%%%%%%%%%%%%%%%%%%%%%%%%%%%%%%%%%%%

% INDICI - ELIMINARE GLI INDICI NON NECESSARI

% INDICE GENERALE
\tableofcontents

% INDICE DELLE FIGURE
\listoffigures

% INDICE DELLE TABELLE
% \listoftables

% INDICE DEI CODICI
% \listofcodes

%%%%%%%%%%%%%%%%%%%%%%%%%%%%%%%%%%%%%%%%%%%%%%%%%%%%%%%%%%%%%%%

\mainmatter

% INCLUSIONE FILE CAPITOLI - PERSONALIZZARE - TENERE COERENTE CON LISTA IN ALTO
\chapter{Introduzione}
\label{chap:Introduzione}


La Blockchain è uno degli argomenti più caldi nell'industria dell'IT. In questo 
capitolo cercheremo di fare chiarezza, partendo dalle origini di questa tecnologia
sino ad arrivare allo stato dell'arte odierno.

\section{DLT}
Per capire cosa è una \textbf{Distribuited Ledger Technology} è necessario 
innanzitutto chiarire il concetto di Ledger.

Un \textit{Ledger}, parola che in italiano traduciamo come \textit{Libro Mastro}, 
è il registro principale in cui sono riunite tutte le transazioni economiche 
che compongono un dato sistema contabile. Si tratta in sostanza di un documento
nel quale vengono registrate tutte le transazioni economiche, dove viene
indicata la data della transazione, la cifra scambiata e i soggetti coinvolti 
nella transazione (chi dà e chi riceve).

\include{chapters/chap_tre}
\chapter{Il Protocollo}

In questo capitolo verrà trattato il protocollo proposto per l'implementazione
della Time-Lock Encryption sui DLT. Nella prima parte cercheremo di spiegare
l'idea che sta alla base, anche con l'ausilio di un esempio. Nella seconda parte
invece passeremo ad analizzare gli aspetti più formali, come le caratteristiche
dello Smart Contract che
serve implementare e i requisiti del DLT sui cui operare.

\section{L'idea}
\subsection{Versione base}
Immaginiamo che Carol abbia bisogno di Time-Lock Encryption su un certo messaggio $ x $.
Per farlo decide farsi aiutare da Alice. Quest'ultima
impegna a conservare il messaggio e ha renderlo pubblico solo dopo un
certo istante di tempo $ \tau $.
Immaginiamo inoltre che venga fissata una certa ricompensa da corrispondere ad Alice
per il suo servizio di conservazione di $ x $.

Alice vuole essere sicura di
ottenere la ricompensa se rispetta il suo impegno. Allo stesso tempo Carol
vuole avere la certezza che Alice possa riscattare il premio solo se si comporta
in maniera corretta.
Carol inoltre non vuole che soggetti terzi partecipino all'accordo, perché desidera
che sia $ x $ sia l'accordo stesso rimangano segreti.
Per farlo decidono di usare uno smart contract.

Nella fase iniziale Carol invia ad Alice il messaggio $ x $. Allo stesso tempo invia
allo smart contract una certa cifra in criptomoneta
che corrispone alla somma tra il premio \textit{prize} da corrispondere ad Alice e un
\textit{pawn} che le verrà ritornato al termine delle operazioni, un hash
crittografico del segreto $ x $ e l'istante di tempo $ \tau $.
\begin{figure}[H]
	\centering
	\includegraphics[width=0.3\linewidth]{images/chap_protocollo/step-0.pdf}
	\caption{Step 0}
\end{figure}


Alice si impegna quindi a conservare il segreto, e al tempo $ \tau $ lo rende noto.
Per farlo invia allo smart contract $ x $, ed in cambio ottiene il suo premio
\textit{price}.
Allo stesso tempo, Alice riottiene il \textit{pawn} che aveva versato in precedenza.
\footnote{Ad una prima analisi può sembrare che il \textit{pawn} sia inutile, 
ma in realtà è necessario per proteggersi da alcuni tipi di attacchi.
Le ragioni dettagliate verranno discusse nei capitoli successivi.}
\begin{figure}[H]
	\centering
	\includegraphics[width=0.3\linewidth]{images/chap_protocollo/step-ok.pdf}
	\caption{Step ok}
\end{figure}

Cosa succede se Alice prova a riscattare il premio prima dell'istante $ \tau $?
Semplicemente lo smart contract rifiuta la sua richiesta.
\begin{figure}[H]
	\centering
	\includegraphics[width=0.3\linewidth]{images/chap_protocollo/step-anticipo.pdf}
	\caption{Step anticipo}
\end{figure}

E se Carol cedesse il segreto ad Eve prima del tempo $ \tau $?
In questo caso Eve usare $ x $ per ottenere un \textit{counterprize}. 
Ciò inoltre impedirebbe ad Alice di ottenere il suo premio. 
È evidente che l'interesse di Alice sia quello di mantenere $ x $ segreto
\begin{figure}[H]
	\begin{minipage}{0.5\textwidth}
        \centering
        \includegraphics[width=.9\linewidth]{images/chap_protocollo/leak-1.pdf}
        \caption{Leak 1}
      \end{minipage}\hfill
      \begin{minipage}{0.5\textwidth}
        \centering
        \includegraphics[width=.9\linewidth]{images/chap_protocollo/leak-2.pdf}
        \caption{Leak 2}
      \end{minipage}
\end{figure}


\subsection{Versione avanzata}
Facciamo notare che nello scenario di cui sopra Alice conosce sin 
dall'inizio il messaggio
$ x $, perché le è stato affidato nella prima fase del processo.
Ma se Carol volesse che il messaggio rimanga segreto anche ad Alice?
Per fare ciò Alice ha bisogno di (almeno) un altro collaboratore, Bob, e di un
algoritmo di \textbf{secret sharing}, \footnote{Un algoritmo di secret sharing
è un algoritmo che permette di
distribuire un certo segreto tra un gruppo di partecipanti, ad ognuno dei quali viene
assegnato uno \textit{share}. Il segreto può essere ricostruito solo unendo un certo
numero di share.}


\chapter{Proof of Concept}

In questo capitolo costruiremo una soluzione sullo Stellar Network
compatibile con il protocollo proposto al capitolo \ref{chap:protocollo}.
Seguirà una Proof of Concept di quanto descritto.

Avremmo potuto proporre una
implementazione anche su altri DLT, come ad esempio Ethereum. Abbiamo scelto
Stellar perché ci da modo di dimostrare che non è necessario un ambiente che
offre degli smart contract "potenti" come nel caso della la Ethereum Virtual Machine,
ma è sufficiente un supporto
"debole" agli smart contract (\ref{subsec:stellar smart contracts}).


\section{Un'implementazione sullo Stellar Network}
Sia $ U_0 $ l'utente che richiede il servizio di Time-Lock Encryption,
siano $ U_1, ...\, , U_N $ gli utenti che offrono il servizio di conservazione e
siano rispettivamente $ S_1, ...\, , S_N $ i segreti assegnati agli utenti
(share o messaggi).

Per prima cosa $ U_0 $ crea gli account $ A_1, ...\, , A_N $ e
versa su questi account una cifra pari alla somma tra
il \textit{prize}, il \textit{pawn} e il costo che dovrà sostenere per le commissioni
delle transazioni che dovrà effettuare.
TODO SPIEGARE COME CALCOLARE IL COSTO DELLE COMMISSIONI.

A questo punto per ogni account $ A_i $ crea $ N $ transazioni
ognuna delle quali rivolta ad uno degli utenti $ U_1,\ ...\ , U_N $, per un totale
di $ N \times N $ transazioni. Indichiamo con $ T_{{A_i}{U_j}} $ la transazione
proveniente dall'account $ A_i $ e diretta all'utente $ U_j $.
Per semplicità di notazione poniamo $ T_{{A_i}{U_j}} = T_{ij} $.

Le transazioni sono così fatte:
\begin{itemize}
	\item $T_{ii} $ \textbf{Transazione Prize}
	      \begin{itemize}
		      \item SEQUENCE\textunderscore NUMBER = nextSequenceNumber($ A_i $) + 1
		      \item MIN\textunderscore TIME = $ \tau $
		      \item MAX\textunderscore TIME = $ \tau + \delta $
		      \item OPERATIONS
		            \begin{enumerate}
			            \item SEND \textit{prize} TO $ U_i $
			            \item SEND \textit{pawn} TO $ U_0 $
		            \end{enumerate}
	      \end{itemize}
\end{itemize}

\begin{itemize}
	\item $T_{ij} $, con $ i \neq j $ \textbf{Transazione Counterprize}
	      \begin{itemize}
		      \item SEQUENCE\textunderscore NUMBER = nextSequenceNumber($ A_i $) + 1
		      \item MIN\textunderscore TIME = \textit{None}
		      \item MAX\textunderscore TIME = $ \tau $
		      \item OPERATIONS
		            \begin{enumerate}
			            \item SEND \textit{counterprize} TO $ U_j $
		            \end{enumerate}
	      \end{itemize}
\end{itemize}

Facciamo notare che tutte queste transazioni hanno lo stesso sequence number.
Ciò significa che al più una di queste potrà essere eseguita.
\\
\\
Ora $ U_0 $ crea ulteriori $ N $ transazioni, una per ogni account,
con queste caratteristiche:
\begin{itemize}
	\item $T_{i0} $ \textbf{Transazione d'inizializzazione}
	      \begin{itemize}
		      \item SEQUENCE\textunderscore NUMBER = nextSequenceNumber($ A_i $)
		      \item OPERATIONS
		            \begin{enumerate}
			            \item SET \textit{master weight} = 255
			            \item SEND \textit{med threshold} = 2
			            \item SEND \textit{high threshold} = 254
			            \item APPEND PRE\textunderscore AUTH\textunderscore TX
			                  \textit{hash($ T_{ij} $)} WITH WEIGHT 1
			                  \\(per ogni $ T_{ij} $, con $ 1 \leq j \leq N $)
			            \item APPEND HASHX\textunderscore SIGNER
			                  \textit{hash($ S_{i} $)} WITH WEIGHT 1
			            \item SET \textit{master weight} = 255

		            \end{enumerate}
	      \end{itemize}
\end{itemize}

Infine $ U_0 $ fa il submit sul network delle transazioni
d'inizializzazione $ T_{i0} $ e comunica agli utenti il segreto che
devono conservare.





\section{Lo script}
Di seguito uno script scritto in Python che applica quanto detto sopra.
Si fa uso del \textit{Python SDK per Stellar} e di
\textit{secretsharing di Blockstack}, un'implementazione open-source dello
Shamir Secret Sharing.
\lstinputlisting[caption={}]{code/poc/main.py}
Github:
\url{https://github.com/imcatta/stellar-time-lock-encryption-poc}
\chapter{Analisi dei possibili attacchi}
\label{chap:analisi-attacchi}

In questo capitolo analizzeremo i possibili attacchi a cui il protocollo proposto
è soggetto.
\section{Comportamenti malevoli dei signoli utenti}
\subsection{Comportamento malevole del client}
\subsection{Comportamento malevolo di un provider}


\section{Coalizioni tra provider}
\subsection{Coalizione tra tutti i providers}
\subsection{Coalizione tra $ N $ providers}

\section{Coalizioni tra client e providers}

\section{Un singolo attore controlla più providers}

\section{Un soggetto corrompe i providers}
\chapter{Scelta dei parametri del protocollo}
\label{chap:dimensionamento}

In questo capitolo vedremo quali sono i parametri del protocollo
e analizzaremo alcuni criteri per il dimensionamento.

\section{I Parametri}

\subsection{La deadline $ \tau $}
La deadline $ \tau $ rappresenta l'istante di tempo in cui
si desidera che il segreto venga reso pubblico.
Si tratta di un parametro deciso a priori, sul quale difficilmente si riesce ad
intervenire. Ad ogni modo è bene sapere che più $ \tau $ è lontano, più aumentano
le possibilità che qualcosa vada storto. La ragione è che più è lontana la deadline,
più cresce la probabilità che i \textit{provider} smarriscano o vengano derubati
del segreto.

\subsection{La tolleranza $ \delta $}
La tolleranza $ \delta $ rappresenta quanto ritardo rispetto a $ \tau $ è
accettato per la pubblicazione del segreto. Una tolleranza troppo
piccola potrebbe causare problemi se il client non è abbastanza reattivo
al tempo $ \tau $, ad esempio
per un intasamento del network o
perché è
temporaneamente offline.
Allo stesso tempo una tolleranza
troppo grande potrebbe causare un ritardo non accettabile rispetto alla deadline.
Per la scelta di questo parametro è ben tenere in considerazione anche
la velocità di elaborazione delle transazioni
del DLT sottostante.

\subsection{Il \textit{prize}}
Il \textit{prize} è la ricompensa da corrispondere al provider per il suo servizio
di detenzione del segreto. È una cifra che deve essere concordata tra le parti,
ed è bene che sia proporzionale a quanto dista $ \tau $ dal momento
dell'inizializzazione del protocollo
e a quanto è importante il segreto.
$$ \textit{prize} \propto \tau - t_{init} $$

\subsection{Il \textit{counterprize}}
\label{subsec:counterprize}
Il \textit{counterprize} è la cifra che viene corrisposta a chi invia il segreto
allo smart contract prima del tempo $ \tau $. È bene che
\begin{center}
	\textit{prize} $ \gg $ \textit{counterprize}
\end{center}
Il motivo di questo vincolo è che il provider potrebbe in un qualsiasi istante
prima di $ \tau $ ottenere
il \textit{counterprize} (perché chiaramente conosce il segreto) e quindi
rendere noto il segreto prima della deadline. Se però il
\textit{prize} è molto più grande sarebbe contro il suo
interesse farlo,
poiché perderebbe la possibilità di ottenere un guadagno significativamente
maggiore al tempo $ \tau $.

\subsection{Il \textit{pawn}}
Il \textit{pawn} è la cifra che viene restituita al client quando il segreto
viene pubblicato nella finestra temporale $ \tau \leq t \leq \tau + \delta $.

Il \textit{pawn} è necessario perché, ovviamente, anche il client conosce il segreto
e quindi potrebbe ottenere il \textit{counterprize} (magari poco prima della deadline).
Se lo facesse il provider non potrebbe più ottenere il \textit{prize} anche se si è
comportato in maniera corretta, venendo in un certo senso truffato dal client.
Come nel caso del \hyperref[subsec:counterprize]{\textit{counterprize}}
si vuole scoraggiare questo comportamento con il vincolo
\begin{center}
	\textit{pawn} $ \gg $ \textit{counterprize}
\end{center}

\subsection{Numero di share}
Il numero di share $ n $ è uno dei parametri
dell'algoritmo di secret sharing, utilizzato dalla versione
avanzata del protocollo.
La scelta di questo valore determina automaticamente anche il numero di provider
necessari.
Facciamo notare che la versione base può essere vista come un caso particolare
della versione avanzata, dove $ n = 1 $.

\subsection{Threshold per la ricostruzione del segreto}
Il threshold $ t $ rappresenta il numero minimo di share che
sono necessari per la ricostruzione del segreto.
Anche in questo caso la versione base può essere vista come un caso particolare
della versione avanzata, dove $ t = 1 $.

\section{Criteri di dimensionamento}

\subsection{Denaro necessario}
Il denaro necessario $ d $ rappresenta quanto denaro è necessario per inizializzare
il protocollo.
$$ d = \sum_{i=1}^{N} \textit{prize}_i + pawn_i + fee_i $$
Nell'ipotesi che tutti i \textit{prize} e tutti i \textit{pawn}
siano uguali tra loro e che le commissioni \textit{fee} siano trascurabili
l'espressione diventa
$$ d = N(prize + pawn) $$
\textbf{N.B.} Questo valore non è il costo che deve sostenere il client, perché parte
di questi soldi gli verranno restituiti attraverso i \textit{pawn}.

\subsection{Costo}
Il costo $ c $ rappresenta quanto denaro dovrà spendere il client per il servizio
richiesto. Il costo non è fissato a priori,
perché dipende da quanti \textit{pawn} verranno
restituiti.
$$ c = \sum_{i=1}^{N} (\textit{prize}_i + \textit{fee}_i) + \sum \textit{pawn}_j $$
\begin{flushright}
	dove $ \sum \textit{pawn}_j $ è la somma dei \textit{pawn} restituiti.
\end{flushright}
Nel caso migliore, ossia quando tutti i \textit{pawn} ritornano al client diventa
$$ c = \sum_{i=1}^{N} (prize_i + fee_i) $$
Mentre nel caso peggiore, ossia quando nessun \textit{pawn}
ritorna al client, è pari a
$$ c = d = \sum_{i=1}^{N} (prize_i + pawn_i + fee_i) $$


\subsection{Resistenza a smarrimenti}
La resistenza agli smarrimenti $ \theta $ rappresenta 
il numero di provider che possono non
rendere noto il proprio share senza compromettere la pubblicazione del messaggio
originale.
$$ \theta = n - t $$
È interessante valutare questo valore in relazione
al numero totale di share $ n $.
$$ \theta_r = \frac{n - t}{n} $$
$$ \theta_\% = 100 \cdot \theta_r = 100 \cdot \frac{n - t}{n} $$

\subsection{Resistenza a furti}
\label{subsec:resistenza-a-furti}
La resistenza ai furti $ \gamma $ è il numero di share che un soggetto deve riuscire
ad ottenere dai provider affinchè possa
ricostruire il segreto prima del tempo $ \tau $.
Questo numero è ovviamente pari al threshold.
\begin{center}
	$ \gamma = t $
\end{center}









\chapter{Possibili usi}

\section{Gara d'appalto a buste chiuse}
\label{sec:gara-buste-chiuse}

% \appendix
% % INCLUSIONE APPENDICI - - PERSONALIZZARE - TENERE COERENTE CON LISTA IN ALTO

%%%%%%%%%%%%%%%%%%%%%%%%%%%%%%%%%%%%%%%%%%%%%%%%%%%%%%%%%%%%%%%

% BIBLIOGRAFIA
\phantomsection
\addcontentsline{toc}{chapter}{\refname}
\nocite{*}
\printbibliography


%%%%%%%%%%%%%%%%%%%%%%%%%%%%%%%%%%%%%%%%%%%%%%%%%%%%%%%%%%%%%%%
% RINGRAZIAMENTI - PERSONALIZZARE
\begin{titlepage}
	\thispagestyle{empty}
	\ringraziamenti
	Desidero innanzitutto ringraziare il professor Stefano Paraboschi per aver accettato
	l’incarico di relatore per la mia tesi.
	Inoltre, ringrazio sentitamente l'ing. Enrico Bacis, l'ing. Dario Facchinetti
	e l'ing. Marco Rosa per il tempo dedicatomi e per la disponibilità
	dimostrata nel redimere i miei dubbi durante la stesura del lavoro.
	Infine, ho
	desiderio di ringraziare con affetto i miei genitori ed i miei amici per il sostegno dato
	e per essermi stati vicino ogni momento durante tutto il percorso universitario.
\end{titlepage}
%%%%%%%%%%%%%%%%%%%%%%%%%%%%%%%%%%%%%%%%%%%%%%%%%%%%%%%%%%%%%%%

\end{document}