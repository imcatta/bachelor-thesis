\chapter{Scelta dei parametri del protocollo}
\label{chap:dimensionamento}

In questo capitolo vedremo quali sono i parametri del protocollo
e analizzaremo alcuni criteri per il dimensionamento.

\section{I Parametri}

\subsection{La deadline $ \tau $}
La deadline $ \tau $ rappresenta l'istante di tempo in cui
si desidera che il segreto venga reso pubblico.
Si tratta di un parametro deciso a priori, sul quale difficilmente si riesce ad
intervenire. Ad ogni modo è bene sapere che più $ \tau $ è lontano, più aumentano
le possibilità che qualcosa vada storto.

\subsection{La tolleranza $ \delta $}
La tolleranza $ \delta $ rappresenta quanto ritardo rispetto a $ \tau $ è
accettato per la pubblicazione del segreto. Una tolleranza troppo
piccola potrebbe causare problemi se ad esempio il provider è
temporaneamente offline al tempo $ \tau $.
Allo stesso tempo una tolleranza
troppo grande potrebbe causare un ritardo non accettabile rispetto alla deadline.
Per la scelta di questo parametro bisogna anche tenere in considerazione
anche la velocità di elaborazione delle transazioni
del DLT sottostante.
Una buona scelta per questo parametro potrebbe essere un valore compreso tra
un minimo di 10 minuti e un massimo di 7 giorni.

\subsection{Il \textit{prize}}
Il \textit{prize} è la ricompensa da corrispondere al provider per il suo servizio
di detenzione del segreto. È una cifra che deve essere concordata tra le parti,
ed è bene che sia proporzionale a quanto dista $ \tau $ dal momento
dell'inizializzazione del protocollo
e a quanto è importante il segreto.

\subsection{Il \textit{counterprize}}
\label{subsec:counterprize}
Il \textit{counterprize} è la cifra che viene corrisposta a chi invia il segreto
allo smart contract prima del tempo $ \tau $. È bene che
\begin{center}
	\textit{prize} $ \gg $ \textit{counterprize}
\end{center}
Il motivo è che il provider potrebbe potenzialmente ottenere
il \textit{counterprize} (perché chiaramente conosce il segreto) e quindi
rendere noto il segreto prima di $ \tau $. Se però il
\textit{prize} è molto più grande sarebbe contro il suo
interesse farlo,
poiché perderebbe la possibilità di ottenere un guadagno significativamente
maggiore al tempo $ \tau $.

\subsection{Il \textit{pawn}}
Il \textit{pawn} è la cifra che viene restituita al client quando il segreto
viene pubblicato nella finestra temporale $ \tau \leq t \leq \tau + \delta $.

Il \textit{pawn} è necessario perché, ovviamente, anche il client conosce il segreto
e quindi potrebbe ottenere il \textit{counterprize} (magari poco prima della deadline).
Se lo facesse il provider non potrebbe più ottenere il \textit{prize} anche se si è
comportato in maniera corretta, venendo in un certo senso truffato dal client.
Come nel caso del \hyperref[subsec:counterprize]{\textit{counterprize}}
si vuole scoraggiare questo comportamento con il vincolo.
\begin{center}
	\textit{pawn} $ \gg $ \textit{counterprize}
\end{center}

\subsection{Numero di share}

\subsection{Threshold per la ricostruzione del segreto}

\section{Criteri di dimensionamento}
\subsection{Costo}
\subsection{Denaro necessario}
\subsection{Resistenza a smarrimenti}
\subsection{Resistenza a furti}








