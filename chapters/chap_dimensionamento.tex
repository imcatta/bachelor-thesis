\chapter{Scelta dei parametri del protocollo}
\label{chap:dimensionamento}

In questo capitolo vedremo quali sono i parametri del protocollo
e analizzeremo alcuni criteri per il dimensionamento.

\section{I Parametri}

\subsection{La deadline $ \tau $}
La deadline $ \tau $ rappresenta l'istante di tempo in cui
si desidera che il segreto venga reso pubblico.
Si tratta di un parametro deciso a priori, sul quale difficilmente si riesce ad
intervenire. Ad ogni modo è bene sapere che più $ \tau $ è lontano, più aumentano
le possibilità che qualcosa vada storto. La ragione è che più è lontana la deadline,
più crescono la probabilità di smarrimento del segreto $ P_s $ e l'appetibilità
del messaggio da parte di eventuali attaccanti (quindi $ P_c $ e $ P_m $).

\subsection{La tolleranza $ \delta $}
La tolleranza $ \delta $ rappresenta quanto ritardo rispetto a $ \tau $ è
accettato per la pubblicazione del segreto. Una tolleranza troppo
piccola potrebbe causare problemi se il client non è abbastanza reattivo
al tempo $ \tau $, ad esempio
per un intasamento del network o
perché è
temporaneamente offline.
Allo stesso tempo una tolleranza
troppo grande potrebbe causare un ritardo non accettabile rispetto alla deadline.
Per la scelta di questo parametro è ben tenere in considerazione anche
la velocità di elaborazione delle transazioni
della blockchain utilizzata.

\subsection{Il \textit{prize}}
Il \textit{prize} è la ricompensa da corrispondere all'agente per il suo servizio
di detenzione del segreto. È una cifra che deve essere concordata tra le parti.
È bene che sia proporzionale a quanto dista $ \tau $ dal momento
dell'inizializzazione del protocollo
e a quanto è importante il segreto.
$$ \textit{prize} \propto \tau - t_{init} $$

\subsection{Il \textit{counterprize}}
\label{subsec:counterprize}
Il \textit{counterprize} è la cifra che viene corrisposta a chi invia il segreto
allo smart contract prima del tempo $ \tau $. È bene che
\begin{center}
	\textit{prize} $ \gg $ \textit{counterprize}
\end{center}
Il motivo di questo vincolo è che l'agente potrebbe ottenere
il \textit{counterprize} in un qualsiasi istante
prima di $ \tau $ (perché chiaramente conosce lo share) e quindi
rendere noto il segreto prima della deadline. Se però il
\textit{prize} è molto più grande del \textit{counterprize} sarebbe contro il suo
interesse fare questa azione,
poiché perderebbe la possibilità di ottenere un guadagno significativamente
maggiore al tempo $ \tau $.

\subsection{Il \textit{pawn}}
\label{subsec:pawn}
Il \textit{pawn} è la cifra che viene restituita al client quando il segreto
viene pubblicato nella finestra temporale $ \tau \leq t \leq \tau + \delta $.

Il \textit{pawn} è necessario perché, ovviamente, anche il client conosce il segreto
e quindi potrebbe ottenere il \textit{counterprize} (magari poco prima della deadline).
Se lo facesse l'agente non potrebbe più ottenere il \textit{prize} anche se si è
comportato in maniera corretta, venendo in un certo senso truffato dal client.
Come nel caso del \hyperref[subsec:counterprize]{\textit{counterprize}}
si vuole scoraggiare questo comportamento con il vincolo
\begin{center}
	\textit{pawn} $ \gg $ \textit{counterprize}
\end{center}

\section{Criteri di dimensionamento}

\subsection{Denaro necessario}
Il denaro necessario $ d $ rappresenta quanto denaro è necessario per inizializzare
il protocollo.
$$ d = \sum_{i=1}^{N} \textit{prize}_i + pawn_i + fee_i $$
Nell'ipotesi che tutti i \textit{prize} e tutti i \textit{pawn}
siano uguali tra loro e che le commissioni \textit{fee} siano trascurabili
l'espressione diventa
$$ d = N(prize + pawn) $$
\textbf{N.B.} Questo valore non è il costo che deve sostenere il client, perché parte
di questi soldi gli verranno restituiti attraverso i \textit{pawn}.

\subsection{Costo}
Il costo $ c $ rappresenta quanto denaro dovrà spendere il client per il servizio
richiesto. Il costo non è fissato a priori,
perché dipende da quanti \textit{pawn} verranno
restituiti.
$$ c = \sum_{i=1}^{N} (\textit{prize}_i + \textit{fee}_i) + \sum \textit{pawn}_j $$
\begin{center}
	dove $ \sum \textit{pawn}_j $ è la somma dei \textit{pawn} restituiti.
\end{center}

Nel caso migliore, ossia quando tutti i \textit{pawn} ritornano al client diventa
$$ c = \sum_{i=1}^{N} (prize_i + fee_i) $$
Mentre nel caso peggiore, ossia quando nessun \textit{pawn}
ritorna al client, è pari a
$$ c = d = \sum_{i=1}^{N} (prize_i + pawn_i + fee_i) $$
