\chapter{Proof of Concept}

In questo capitolo costruiremo una soluzione sullo Stellar Network
compatibile con il protocollo proposto al capitolo \ref{chap:protocollo}.
Seguirà una Proof of Concept di quanto descritto.

Avremmo potuto proporre una
implementazione anche su altri DLT, come ad esempio Ethereum. Abbiamo scelto
Stellar perché ci da modo di dimostrare che non è necessario un ambiente che
offre degli smart contract "potenti" come nel caso della la Ethereum Virtual Machine,
ma è sufficiente un supporto
"debole" agli smart contract (\ref{subsec:stellar smart contracts}).


\section{Un'implementazione sullo Stellar Network}
Sia $ U_0 $ l'utente che richiede il servizio di Time-Lock Encryption,
siano $ U_1, ...\, , U_N $ gli utenti che offrono il servizio di conservazione e
siano rispettivamente $ S_1, ...\, , S_N $ i segreti assegnati agli utenti
(share o messaggi).

Per prima cosa $ U_0 $ crea gli account $ A_1, ...\, , A_N $ e
versa su questi account una cifra pari alla somma tra
il \textit{prize}, il \textit{pawn} e il costo che dovrà sostenere per le commissioni
delle transazioni che dovrà effettuare.
TODO SPIEGARE COME CALCOLARE IL COSTO DELLE COMMISSIONI.

A questo punto per ogni account $ A_i $ crea $ N $ transazioni
ognuna delle quali rivolta ad uno degli utenti $ U_1,\ ...\ , U_N $, per un totale
di $ N \times N $ transazioni. Indichiamo con $ T_{{A_i}{U_j}} $ la transazione
proveniente dall'account $ A_i $ e diretta all'utente $ U_j $.
Per semplicità di notazione poniamo $ T_{{A_i}{U_j}} = T_{ij} $.

Le transazioni sono così fatte:
\begin{itemize}
	\item $T_{ii} $ \textbf{Transazione Prize}
	      \begin{itemize}
		      \item SEQUENCE\textunderscore NUMBER = nextSequenceNumber($ A_i $) + 1
		      \item MIN\textunderscore TIME = $ \tau $
		      \item MAX\textunderscore TIME = $ \tau + \delta $
		      \item OPERATIONS
		            \begin{enumerate}
			            \item SEND \textit{prize} TO $ U_i $
			            \item SEND \textit{pawn} TO $ U_0 $
		            \end{enumerate}
	      \end{itemize}
\end{itemize}

\begin{itemize}
	\item $T_{ij} $, con $ i \neq j $ \textbf{Transazione Counterprize}
	      \begin{itemize}
		      \item SEQUENCE\textunderscore NUMBER = nextSequenceNumber($ A_i $) + 1
		      \item MIN\textunderscore TIME = \textit{None}
		      \item MAX\textunderscore TIME = $ \tau $
		      \item OPERATIONS
		            \begin{enumerate}
			            \item SEND \textit{counterprize} TO $ U_j $
		            \end{enumerate}
	      \end{itemize}
\end{itemize}

Facciamo notare che tutte queste transazioni hanno lo stesso sequence number.
Ciò significa che al più una di queste potrà essere eseguita.
\\
\\
Ora $ U_0 $ crea ulteriori $ N $ transazioni, una per ogni account,
con queste caratteristiche:
\begin{itemize}
	\item $T_{i0} $ \textbf{Transazione d'inizializzazione}
	      \begin{itemize}
		      \item SEQUENCE\textunderscore NUMBER = nextSequenceNumber($ A_i $)
		      \item OPERATIONS
		            \begin{enumerate}
			            \item SET \textit{master weight} = 255
			            \item SEND \textit{med threshold} = 2
			            \item SEND \textit{high threshold} = 254
			            \item APPEND PRE\textunderscore AUTH\textunderscore TX
			                  \textit{hash($ T_{ij} $)} WITH WEIGHT 1
			                  \\(per ogni $ T_{ij} $, con $ 1 \leq j \leq N $)
			            \item APPEND HASHX\textunderscore SIGNER
			                  \textit{hash($ S_{i} $)} WITH WEIGHT 1
			            \item SET \textit{master weight} = 255

		            \end{enumerate}
	      \end{itemize}
\end{itemize}

Infine $ U_0 $ fa il submit sul network delle transazioni
d'inizializzazione $ T_{i0} $ e comunica agli utenti il segreto che
devono conservare.





\section{Lo script}
Di seguito uno script scritto in Python che applica quanto detto sopra.
Si fa uso del \textit{Python SDK per Stellar} e di
\textit{secretsharing di Blockstack}, un'implementazione open-source dello
Shamir Secret Sharing.
\lstinputlisting[caption={}]{code/poc/main.py}
Github:
\url{https://github.com/imcatta/stellar-time-lock-encryption-poc}