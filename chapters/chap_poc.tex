\chapter{Proof of Concept}

In questo capitolo costruiremo una soluzione sullo Stellar Network
compatibile con il protocollo proposto al capitolo \ref{chap:protocollo}.
Seguirà una Proof of Concept di quanto descritto.

Avremmo potuto proporre una
implementazione anche su altri DLT, come ad esempio Ethereum. Abbiamo scelto
Stellar perché ci da modo di dimostrare che per un'implementazione funzionante
del protocollo proposto non è necessario un ambiente che
offre degli smart contract "potenti" come ad esempio
la \textit{Ethereum Virtual Machine},
ma è sufficiente un supporto
"debole" agli smart contract (si veda \ref{subsec:stellar smart contracts}).


\section{Un'implementazione sullo Stellar Network}
Sia $ U_0 $ il \textit{client};
siano $ U_1, ...\, , U_N $ gli \textit{agenti} e
siano rispettivamente $ S_1, ...\, , S_N $ i segreti assegnati agli agenti
(share o messaggi).

Per prima cosa $ U_0 $ crea gli account $ A_1, ...\, , A_N $ e
versa su questi account una cifra pari alla somma tra
il \textit{prize}, il \textit{pawn} e il costo che dovrà sostenere per le commissioni
delle transazioni che dovrà effettuare. Per il calcolo delle commissioni si rimanda
alla sezione \ref{sec:calcolo-commissioni}

A questo punto per ogni account $ A_i $ il client crea $ N $ transazioni
ognuna delle quali rivolta ad uno degli utenti $ U_1,\ ...\ , U_N $, per un totale
di $ N \times N $ transazioni. Indichiamo con $ T_{{A_i}{U_j}} $ la transazione
proveniente dall'account $ A_i $ e diretta all'utente $ U_j $.
Per semplicità di notazione poniamo $ T_{{A_i}{U_j}} = T_{ij} $.

Le transazioni sono così fatte:
\begin{itemize}
	\item $T_{ii} $ \textbf{Transazione Prize}
	      \begin{itemize}
		      \item SEQUENCE\textunderscore NUMBER = nextSequenceNumber($ A_i $) + 1
		      \item MIN\textunderscore TIME = $ \tau $
		      \item MAX\textunderscore TIME = $ \tau + \delta $
		      \item OPERATIONS
		            \begin{enumerate}
			            \item SEND \textit{prize} TO $ U_i $
			            \item SEND \textit{pawn} TO $ U_0 $
		            \end{enumerate}
	      \end{itemize}
\end{itemize}

\begin{itemize}
	\item $T_{ij} $, con $ i \neq j $ \textbf{Transazione Counterprize}
	      \begin{itemize}
		      \item SEQUENCE\textunderscore NUMBER = nextSequenceNumber($ A_i $) + 1
		      \item MIN\textunderscore TIME = \textit{None}
		      \item MAX\textunderscore TIME = $ \tau $
		      \item OPERATIONS
		            \begin{enumerate}
			            \item SEND \textit{counterprize} TO $ U_j $
		            \end{enumerate}
	      \end{itemize}
\end{itemize}

Facciamo notare che tutte queste transazioni hanno lo stesso sequence number.
Ciò comporta che al più una di queste potrà essere eseguita.
\\
\\
Ora $ U_0 $ crea ulteriori $ N $ transazioni, una per ogni account,
con queste caratteristiche:
\begin{itemize}
	\item \label{trans-init:title}
	      $T_{i0} $ \textbf{Transazione d'inizializzazione}
	      \begin{itemize}
		      \item SEQUENCE\textunderscore NUMBER = nextSequenceNumber($ A_i $)
		      \item OPERATIONS
		            \begin{enumerate}
			            \item \label{trans-init:set-master-weight-255}
			                  SET \textit{master weight} = 255
			            \item \label{trans-init:set-med-thres}
			                  SET \textit{med threshold} = 2
			            \item \label{trans-init:set-high-thres}
			                  SET \textit{high threshold} = 254
			            \item \label{trans-init:set-pre-auth-tx}
			                  APPEND PRE\textunderscore AUTH\textunderscore TX
			                  \textit{hash($ T_{ij} $)} WITH WEIGHT 1
			                  \\(per ogni $ T_{ij} $, con $ 1 \leq j \leq N $)
			            \item \label{trans-init:set-hashx-signer}
			                  APPEND HASHX\textunderscore SIGNER
			                  \textit{hash($ S_{i} $)} WITH WEIGHT 1
			            \item \label{trans-init:set-master-weight-0}
			                  SET \textit{master weight} = 0

		            \end{enumerate}
	      \end{itemize}
\end{itemize}
A $ U_0 $ non resta che inviare ad ogni utente $ U_i $\footnote{con $ 1 \leq i \leq N $}
il segreto $ S_i $, le transazioni prize e counterprize
$ T_{ij} $ e fare il submit delle
transazioni d'inizializzazione $ T_{i0} $.
\\
\\
Analizziamo nel dettaglio la \hyperref[trans-init:title]{transazione d'inizializzazione}.
Il \textit{med threshold} è la soglia da raggiungere per effettuare
l'invio di Lumen dall'account.
Con l'operazione \ref{trans-init:set-med-thres} lo abbiamo impostato a 2.
Con le operazioni \ref{trans-init:set-pre-auth-tx} abbiamo pre-autorizzato
le transazioni
$ T_{ij} $ con $ 1 \leq j \leq N $ con peso pari a 1.
Con l'operazione \ref{trans-init:set-hashx-signer} abbiamo aggiunto il segreto
$ S_i $ tra le firme autorizzate, con peso pari a 1.
Infine con le operazioni \ref{trans-init:set-high-thres} e
\ref{trans-init:set-master-weight-0} abbiamo impedito all'utente $ U_0 $ (il
detentore della masterkey) di effettuare alcun tipo di operazione dall'account $ A_i $.
L'operazione \ref{trans-init:set-master-weight-255} ci è servita solo per
non avere problemi
nell'effettuare le operazioni successive.
\\
\\
Lo scenario finale è che l'account $ A_i $ è bloccato: neanche $ U_0 $, il detentore
della master key, può apportarvi modifiche o prelevare fondi. Le uniche operazioni
autorizzate sono le $ T_{ij} $, ossia le transazioni per ottenere il \textit{prize}
o il \textit{counterprize}. Per fare il submit di queste transazioni è inoltre
necessario conoscere $ S_i $, perché in questo modo si riesce a raggiungere
il med threshold.
\section{Calcolo delle commissioni}
\label{sec:calcolo-commissioni}
Per il calcolo delle commissioni dobbiamo tenere in considerazione due aspetti.
\paragraph{Transaction Fee} Le commissioni per transazione si calcolano con la formula
\begin{center}
	\textit{\# of operations} $ \times $ \textit{base fee}
\end{center}
La \textit{base fee} è di 100 stroops (0.00001 XLM).


\section{Limitazioni della soluzione proposta}

\subsection{Numero massimo di firme per transazione}
Una limitazione imposta dallo Stellar Network è che con
una singola transazione è possibile aggiungere al massimo 20 firme ad un account.
Ciò comporta che se $ N >= 20 $ è necessario creare più di una transazione
d'inizializzazione per ogni account.
In generale, il numero delle transazioni d'inizializzazione necessarie è
\begin{center}
	$ \# T_{i0} = \left\lfloor\dfrac{ N + 1 }{ 20 }\right\rfloor $
\end{center}

\subsection{Riscatto dei \textit{counterprize}}
L'implementazione proposta si basa sul fatto che in Stellar è possibile pre-autorizzare
delle transazioni. Uno dei problemi è che non è possibile pre-autorizzare delle transazioni con
un destinatario generico, ma è necessario definirlo a priori. Nel nostro caso ciò comporta che
i possibili destinatari dei \textit{counterprize} devono essere decisi già nella
fase di inizializzazione. Per il tipo di soluzione proposta non vi è un limite teorico
per il numero di transazioni \textit{counterprize} che si possono emettere, ma chiaramente
non è sostenibile l'emissione di transazioni verso tutti i potenziali account Stellar.
Si è deciso quindi di emettere transazioni \textit{counterprize} riscattabili solo dagli
agenti coinvolti.

Ciò comporta che solo gli agenti possono riscattare il \textit{counterprize}. Questo potrebbe
essere un problema, perché significa che l'agente non avrebbe il disincentivo economico
nel cedere il proprio share ad un soggetto esterno. Di seguito proponiamo un protocollo
in grado di risolvere questa situazione.

Immaginiamo che Eve ottenga lo share dall'agente Alice.
Qyello che può fare Eve è decidere di dividersi
il \textit{counterprize} con l'agente Bob. Per farlo Bob pre-autorizza due transazioni

\begin{itemize}
	\item $ T_1 $
	      \begin{itemize}
		      \item SEQUENCE\textunderscore NUMBER = nextSequenceNumber($ A_B $) + 1
		      \item OPERATIONS
		            \begin{enumerate}
			            \item SEND (\textit{counterprize} / 2) TO $ U_j $
			            \item SET \textit{master weight} = 1
		            \end{enumerate}
	      \end{itemize}
\end{itemize}
\begin{itemize}
	\item $ T_2 $
	      \begin{itemize}
		      \item SEQUENCE\textunderscore NUMBER = nextSequenceNumber($ A_B $) + 1
		      \item MIN\textunderscore TIME = $ \bar{t} $
		      \item MAX\textunderscore TIME =  \textit{None}
		      \item OPERATIONS
		            \begin{enumerate}
			            \item SET \textit{master weight} = 1
		            \end{enumerate}
	      \end{itemize}
\end{itemize}

e inizializza lo scambio con
\begin{itemize}
	\item $ T_0 $
	      \begin{itemize}
		      \item SEQUENCE\textunderscore NUMBER = nextSequenceNumber($ A_B $)
		      \item OPERATIONS
		            \begin{enumerate}
			            \item SEND \textit{all} TO $ A_B^{\prime} $
			            \item APPEND PRE\textunderscore AUTH\textunderscore TX
			                  \textit{hash($ T_{1} $)} WITH WEIGHT 1
			            \item APPEND PRE\textunderscore AUTH\textunderscore TX
			                  \textit{hash($ T_{2} $)} WITH WEIGHT 1
			            \item SET \textit{master weight} = 0
		            \end{enumerate}
	      \end{itemize}
\end{itemize}

Quello che avviene con la transazione $ T_0 $ e che
Bob svuota l'account destinatario del \textit{counterprize} $ A_B $, si esclude
dagli aventi diritto di firma su $ A_B $ e allo stesso tempo pre-autorizza
le due transazioni $ T_1 $ e $ T_2 $.

La transazione $ T_1 $ invia ad Eve metà del \textit{counterprize}
e ripristina la firma di Bob su $ A_B $.
L'idea è che sia Eve a fare il submit di questa transazione. L'unico modo che ha per
farlo è quello di fare il submit della transazione \textit{counterprize}, in modo da avere
sull'account $ A_B $ un deposito di fondi sufficienti.

Infine $ T_2 $ serve a tutelare Bob; se Eve non fa il submit di $ T_{1} $ in un tempo ragionevole
$ \bar{t} $ Bob può usare $ T_2 $ per riprendere il possesso del suo account $ A_B $.


\section{Lo script}
Di seguito uno script scritto in Python che applica quanto detto sopra.
Si fa uso del \textit{Python SDK per Stellar} e di
\textit{secretsharing di Blockstack}, un'implementazione open-source dello
Shamir Secret Sharing \cite{Shamir:1979:SS:359168.359176}.

Simuliamo la \textit{versione avanzata}, con Carol che genera due share e li
affida ad Alice e Bob.
Simuliamo inoltre che Alice venga in possesso dello share di Bob, e quindi riesca ad
ottenere il \textit{counterprize} associato.
\\
\inputpython{code/poc/main.py}{0}{999}
\url{https://github.com/imcatta/stellar-timed-release-encryption-poc}