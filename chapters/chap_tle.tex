\chapter{Time-Lock Encryption}

\section{Cosa è la TLE}
La \textbf{Time-Lock Encryption} (TLE) (o \textbf{Timed-Release Encryption})
è un metodo per cifrare un messaggio in modo che possa
essere decifrato solo dopo che una certa deadline è stata raggiunta.
Un avversario, anche se ha a disposizione una grande quantità di potenza di calcolo,
non deve essere in grado di ottenere il messaggio prima della deadline.
Inoltre, dopo la deadline i destinatari devono poter leggere il messaggio,
indipendentemente dalla potenza di calcolo che ha a disposizione e
senza dover interagire con il mittente,
con altri riceventi o con terze parti fidate.

Questo tipo di crittografia è molto difficile da ottenere con un modelli
di computazione standard (come la macchina di Turing), perché questi non mettono
a disposizione un "orologio del mondo reale" affidabile.
Nel capitolo \ref{chap:protocollo} proponiamo una soluzione che si appoggia sui DLT,
i quali mettono a disposizione
le buone proprietà di cui necessita la Time-Lock Encryption.

Il problema dell'\textit{invio di un messaggio nel futuro} è stato proposto per la prima volta
da May \cite{May:time-released-crypto} nel 1993 e
successivamente studiato da Rivest, Shamir and Wagner \cite{Rivest96time-lockpuzzles}.

La Time-Lock Encryption ha diverse applicazioni nel mondo reale. Queste verranno
discusse nel dettaglio nel capitolo \ref{chap:possibili-usi}.

\section{Requisiti}
Ricapitolando, un buon sistema di TLE deve soddisfare i seguenti requisiti.

\subsection{Certezza di pubblicazione del messaggio al tempo $ \tau $}
Un buon sistema di TLE deve garantire che il messaggio venga pubblicato al

\subsection{Segretezza del messaggio sino al tempo $ \tau $}
\label{subsec:segretezza-tle}


\section{Come implementare la TLE}
Sostanzialmente esistono due strategie
per affrontare questo problema

\subsection{Time-Locked Puzzle}
I Time-Locked puzzle sono stati introdotti per la prima volta
da Merkle \cite{Merkle:1978:SCO:359460.359473} e approfonditi
da Bellare et. al. \cite{Bellare:1996:EKE:888619}
e da Rivest et. al. \cite{Rivest96time-lockpuzzles}.
L'idea che sta alla base è che il messaggio venga trasformato in un certo instante $ t_0 $
in modo tale che ogni macchina
(seriale o parallela) debba lavorare per un certo quantitativo di tempo $ \Delta $ per risolvere
il problema computazionale sottostante (puzzle). La somma tra $ t_0 $ e $ \Delta $ è la deadline.


Sebbene rappresenti un approccio elegante nell'ambito della teoria della complessità computazionale,
l'uso \textit{Time-Locked Puzzle} è poco sostenibile nella pratica poiché è poco flessibile e richiede l'uso di
grandi quantità di risorse di calcolo. Inoltre è possibile garantire solo un limite inferiore
sull'istante di pubblicazione del messaggio, dato che la macchina risolutrice potrebbe per vari motivi
non iniziare immediatamente
a lavorare, potrebbe effettuare delle pause durante l'elaborazione
o peggio ancora potrebbe decidere di non risolvere il puzzle.
Un altro problema è che in futuro potrebbero venir creati nuovi strumenti hardware in grado
di ridurre il tempo necessario per la risoluzione del puzzle, qualsiasi esso sia.


Although elegant in the complexity theoretic sense, the time-lock puzzle approach is impractical,
consumes a large amount of computational resources and lacks flexibility. Since
time-lock puzzles try to make “CPU time” and “real time” agree as closely as possible, it can
only solve the relative time problem (with reference to the start of solving the puzzle) with an
approximately controllable time (different machines work at different speeds) and the puzzle
does not automatically become solvable at a given time (if solving is not started immediately
upon receipt). If absolute and precise timing of information release is essential, like the sealed
bid scenario mentioned above, the approach based on a trusted time server is inevitable.
% Alice crittografa il suo messaggio in modo che Bob debba eseguire una computazione non
% paralelizzabile senza mai fermarsi per un certo periodo di tempo per poterlo decriptare.
% Se Alice prevede con precisione la potenza di calcolo di cui disporra Bob da qua
% all'istante $ \tau $, allora Bob riesce a leggere il messaggio all'istante desiderato.

% % L'idea è quella di crittografare il segreto attraverso un problema matematico,
% % risolvibile con un calcolatore, che richiede un certo tempo $ \tau $ stimabile
% % per poter essere risolto.
% % ... inserire algoritmo ...
% % Le criticità di questo approccio sono diverse. In primo luogo vi è la difficoltà nel
% % trovare un algoritmo che soddisfi i requisiti qua sopra, e soprattutto che sia robusto
% % anche a fronte del miglioramento della potenza di calcolo dei computer del futuro.

% % Un'altra questione è necessario che vi sia qualcuno disposto a risolvere il problema

\subsection{Trusted agents (agenti fidati)}
\label{subsec:trusted-agents}
Alice crittografa il messaggio in modo che Bob necessiti di altri valori segreti,
pubblicati da uno o più \textit{trusted agents} al tempo $ \tau $ per poter
decriptare il messaggio. Una volta che gli agenti hanno pubblicato le informazioni,
Bob può leggere il messaggio.