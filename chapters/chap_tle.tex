\chapter{Time-Lock Encryption}

\section{Cosa è la TLE}
Lo scopo della \textbf{Time-Lock Encryption} è quello di "inviare un messaggio nel futuro".

\section{Requisiti}
Quello che ci aspettiamo da un buon sistema di TLE è che abbia le seguenti proprietà.

\subsection{Certezza di pubblicazione del messaggio al tempo $ \tau $}

\subsection{Segretezza del messaggio sino al tempo $ \tau $}
\label{subsec:segretezza-tle}


\section{Come implementare la TLE}
Ipotizziamo che Alice voglia inviare un messaggio a Bob in modo che Bob
non lo possa leggere prima di un certo tempo $ \tau $.
Sostanzialmente esistono due strategie
per affrontare questo problema

\subsection{Time-Lock Puzzle}
Alice crittografa il suo messaggio in modo che Bob debba eseguire una computazione non
paralelizzabile senza mai fermarsi per un certo periodo di tempo per poterlo decriptare.
Se Alice prevede con precisione la potenza di calcolo di cui disporra Bob da qua
all'istante $ \tau $, allora Bob riesce a leggere il messaggio all'istante desiderato.

% L'idea è quella di crittografare il segreto attraverso un problema matematico,
% risolvibile con un calcolatore, che richiede un certo tempo $ \tau $ stimabile
% per poter essere risolto.
% ... inserire algoritmo ...
% Le criticità di questo approccio sono diverse. In primo luogo vi è la difficoltà nel
% trovare un algoritmo che soddisfi i requisiti qua sopra, e soprattutto che sia robusto
% anche a fronte del miglioramento della potenza di calcolo dei computer del futuro.

% Un'altra questione è necessario che vi sia qualcuno disposto a risolvere il problema

\subsection{Trusted agents (agenti fidati)}
\label{subsec:trusted-agents}
Alice crittografa il messaggio in modo che Bob necessiti di altri valori segreti,
pubblicati da uno o più \textit{trusted agents} al tempo $ \tau $ per poter
decriptare il messaggio. Una volta che gli agenti hanno pubblicato le informazioni,
Bob può leggere il messaggio.