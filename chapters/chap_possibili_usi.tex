\chapter{Applicazioni della Timed-Release Encryption}
\label{chap:possibili-usi}

In questo capitolo analizzeremo una serie di possibili 
applicazioni della Timed-Release Encryption.

\section{Offerte in aste a buste chiuse}
TODO 

\section{Insider stock trades}
Un insider, ossia una persona che può avere accesso in anticipo a informazioni
riservate sull’attività economica di una società quotata in borsa,
potrebbe essere legalmente obbligato a
prendere impegni in anticipo sull'acquisto di quote della società
sia per mitigare potenziali
abusi di informazioni interne, sia per proteggere l'insider da false
accuse di utilizzo sleale delle informazioni.
In certe circostanze
si può volere che questi impegni rimangano segreti sino a poco prima
della loro esecuzione. Chiaramente un impegno che non garantisce la non ripudiabilità
non è sufficiente in quanto un insider può creare in anticipo un impegno nascosto
e poi rifiutarsi di renderlo pubblico nel caso in cui l'evento non sia per lui conveniente.

Se un insider cifra in anticipo la sua transazione con la Timed-Release Encryption
può sempre essere legalmente obbligato a eseguire la transazione e allo stesso tempo
i dettagli della transazione rimangono segreti fino al tempo desiderato.
In \cite{rabin2006time} Rabin e Thorpe propongono un protocollo nel quale gli insider
cifrano le loro transazioni in anticipo usando la Timed-Release Encryption.
Queste direttive possono essere per l'acquisto, per la vendita o non fare nulla. In questo
modo l'insider non rivela informazioni al mercato.

\section{Raccolta dati in trial clinici}
Al fine di preservare l'integrità dei trial clinici, le informazioni raccolte durante il trial
posso essere cifrate con la Timed-Release Encryption.
Poiché molti di questi studi sono finanziati da società che sono in grado di guadagnare
o perdere ingenti somme di denaro a seconda del loro esito,
potenzialmente vi possono essere pressioni per ottenere un risultato positivo.
L'uso della Timed-Release Encryption può mitigare bias senza rendere note informazioni
confidenziali riguardo lo studio prima che sia completato. La Timed-Release Encryption
previene le frodi da parte di scienziati non etici, e protegge gli scienziati onesti
da false accuse di frode o pressioni da parte dei loro finanziatori per ottenere un certo risultato.

In un contesto, i dati raccolti dagli scienziati verrebbero cifrati
immediatamente mentre vengono raccolti dai dispositivi medici. In questo modo gli scienziati
non sarebbero in grado di conoscere i dati prima della conclusione di una certa fase dello studio;
ciò previene che i trend che si verificano nelle prime fasi dello studio influenzino
le successive operazioni di raccolta dati.
In un altro contesto, i dati clinici verrebbero forniti immediatamente agli scienziati
e successivamente ad una commissione di auditing tramite Timed-Release Encryption.
In questo modo gli scienziati non comunicano immediatamente alla commissione i dati raccolti
e allo stesso tempo al termine della sperimentazione
la commissione di auditing ha la garanzia che i dati non siano stati alterati.

\section{Voto elettronico}
In alcune forme di votazioni si desidera che non vi siano pubblicazioni di risultati intermedi,
perché potrebbero influenzare gli altri elettori. Se i voti vengono cifrati con
la Timed-Release Encryption durante la votazione, i risultati possono rimanere interamente segreti
fino alla chiusura delle urne.

% In one setting, scientists’ data collection process uses our Service to encrypt
% data directly as they are being collected, for example, by diagnostic devices
% or computer user interfaces. The scientists would not be able to see the data
% collected until the conclusion of a phase of the study; this prevents observations
% of trends in early data collection from affecting future data collection practices.
% In another setting, clinical data would be provided to the scientists in raw
% form immediately and to an auditing board encrypted via time-lapse cryptography.
% The scientists would preserve the confidentiality of their data during the
% study to prevent leaking of information by the auditing board, but would know
% that any tampering with results would be discovered after the expiration of the
% time-lapse.

% \cite{Rivest96time-lockpuzzles}
% What are the applications of timed-release crypto"? Here are a few possibilities (some
% due to May):
% A bidder in an auction wants to seal his bid so that it can only be opened after the
% bidding period is closed.
% A homeowner wants to give his mortgage holder a series of encrypted mortgage payments.
% These might be encrypted digital cash with different decryption dates, so that
% one payment becomes decryptable (and thus usable by the bank) at the beginning of
% each successive month.
% An individual wants to encrypt his diaries so that they are only decryptable after fty
% years.


% % altro paper
% Bids in sealed-bid auctions. Our original motivation for this work came
% from joint work with David Parkes and Stuart Shieber on cryptographic auctions
% 	[17]. In our auction protocol, we realized the need for bidders to issue

% \section{Gara d'appalto a buste chiuse}
% \label{sec:gara-buste-chiuse}