\chapter{Applicazioni della Timed-Release Encryption}
\label{chap:possibili-usi}

In questo capitolo presenteremo un elenco non esaustivo di possibili
applicazioni della Timed-Release Encryption.

\section{Offerte in aste a buste chiuse}
Spesso per l'assegnazione di bandi pubblici si ricorre all'uso di
offerte a buste chiuse. Il funzionamento è il seguente.
Vengono fissati i requisiti minimi che devono essere soddisfatti
dalle offerte, i criteri per l'attribuzione del punteggio alle offerte e una scadenza.
A questo punto ogni partecipante presenta all'ente pubblico
la propria offerta in forma cartacea
all'interno di una busta sigillata, rimanendo
completamente all'oscuro sulla presenza di eventuali altri partecipanti e sul
contenuto delle eventuali offerte presentate. Una volta raggiunta la scadenza
le buste vengono aperte, validate, vengono calcolati i punteggi
e viene decretato il vincitore.

I funzionari pubblici coinvolti hanno un ruolo chiave nel processo.
Sono loro a garantire la segretezza e l'integrità
delle offerte fino alla scadenza del bando.
Inoltre devono garantire il rispetto della scadenza per la presentazione delle offerte.
In questo processo talvolta si verificano casi di corruzione.

Una possibile alternativa è l'uso della Timed-Release Encryption. Ogni partecipante
cripta la propria offerta con la TRE, fissando come deadline la scadenza del bando.
Raggiunta la scadenza le offerte vengono automaticamente rese pubbliche. A differenza
del processo di cui sopra, con la TRE l'integrità
e la segretezza delle offerte vengono garantite da proprietà crittografiche e non dall'uomo.

\section{Insider stock trade}
Un insider, ossia una persona che può avere accesso in anticipo a informazioni
riservate sull’attività economica di una società quotata in borsa,
potrebbe essere legalmente obbligato a
prendere impegni in anticipo sull'acquisto di quote della società
sia per mitigare potenziali
abusi di informazioni interne, sia per proteggere l'insider da false
accuse di utilizzo sleale delle informazioni.
In certe circostanze
si può volere che questi impegni rimangano segreti sino a poco prima
della loro esecuzione. Chiaramente un impegno che non garantisce la non ripudiabilità
non è sufficiente in quanto un insider può creare in anticipo un impegno nascosto
e poi rifiutarsi di renderlo pubblico nel caso in cui l'evento non sia per lui conveniente.

Se un insider cifra in anticipo la sua transazione con la Timed-Release Encryption
può sempre essere legalmente obbligato a eseguire la transazione e allo stesso tempo
i dettagli della transazione rimangono segreti fino al tempo desiderato.
In \cite{rabin2006time} Rabin e Thorpe propongono un protocollo nel quale gli insider
cifrano le loro transazioni in anticipo usando la Timed-Release Encryption.
Queste direttive possono essere per l'acquisto, per la vendita o non fare nulla. In questo
modo l'insider non rivela informazioni al mercato.

\section{Raccolta dati in trial clinici}La Timed-Release Encryption può essere utile per garantire l'integrità dei trial clinici.
Poiché molti di questi studi sono finanziati da società che sono in grado di guadagnare
o perdere ingenti somme di denaro a seconda del loro esito,
potenzialmente vi possono essere pressioni sul gruppo di ricerca per ottenere un risultato positivo.

Il protocollo proposto ha due caratteristiche utili in questo caso d'uso. La \textit{proof of existence in time}, ossia è in grado di dimostrare l'esistenza del messaggio al momento dell'inizializzazione; la \textit{non ripudiabilità}, ossia non è possibile in alcun modo impedire il rilascio del messaggio dopo l'inizializzazione ed è sempre dimostrabile chi sia l'autore del messaggio.

L'idea è quella di chiedere ai ricercatori di cifrare i dati con il protocollo proposto man mano che questi vengono raccolti, fissando come deadline la data di termine della raccolta dati. In questo modo raggiunta la deadline i dati diventano disponibili alla commissione di valutazione. L'assunzione è che i dati raccolti acquisiscono significato solo quando raggiungono un certo volume, ossia solo quando ci si avvicina al termine della raccolta dati. Un ricercatore disonesto nelle ultime fasi della raccolta dati potrebbe cercare di manomettere lo studio, generando dati pre-datati o omettendone alcuni. Se però ha cifrato i dati con il protocollo proposto non può farlo, a causa delle due caratteristiche di cui sopra. Cifrando i dati raccolti con il protocollo proposto i ricercatori onesti si tutelano da false accuse e allo stesso tempo i ricercatori disonesti sono impossibilitati dal manomettere a posteriori i dati raccolti.


\section{Voto elettronico}
In alcune forme di votazioni si desidera che non vi siano pubblicazioni di risultati intermedi,
perché potrebbero influenzare gli altri elettori. Se i voti vengono cifrati con
la Timed-Release Encryption durante la votazione, i risultati possono rimanere interamente segreti
fino alla chiusura delle urne.
