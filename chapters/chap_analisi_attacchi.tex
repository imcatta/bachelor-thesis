\chapter{Analisi di robustezza}
\label{chap:analisi-robustezza}
In questo capitolo presenteremo una tecnica per valutare della robustezza delle
implementazioni del protocollo proposto. L'obiettivo è quello di trovare
la probabilità che il segreto venga pubblicato una volta raggiunta la deadline
(requisito di \hyperref[parag:certezza-pubblicazione]{\textit{certezza di pubblicazione}})
e la probabilità che il messaggio rimanga segreto fino alla deadline
(requisito di \hyperref[parag:segretezza-tre]{\textit{segretezza del messaggio}}).

\section{Terminologia}

\paragraph{Numero di share}
Il numero di share $ n $ è uno dei parametri
dell'algoritmo di secret sharing, utilizzato dalla versione
avanzata del protocollo.
La scelta di questo valore determina automaticamente anche il numero di agenti
necessari.
Facciamo notare che la versione base può essere vista come un caso particolare
della versione avanzata, dove $ n = 1 $.

\paragraph{Threshold per la ricostruzione del segreto}
Il threshold $ t $ rappresenta il numero minimo di share che
sono necessari per la ricostruzione del segreto.
Anche in questo caso la versione base può essere vista come un caso particolare
della versione avanzata, dove $ t = 1 $.

\paragraph{Resistenza a smarrimenti}
La resistenza agli smarrimenti $ \theta $ rappresenta
il numero di agenti che possono non
rendere noto il proprio share senza compromettere la pubblicazione del messaggio
originale.
$$ \theta = n - t $$
È interessante valutare questo valore in relazione
al numero totale di share $ n $.
$$ \theta_r = \frac{n - t}{n} $$
$$ \theta_\% = 100 \cdot \theta_r = 100 \cdot \frac{n - t}{n} $$

\paragraph{Resistenza a furti}
La resistenza ai furti $ \gamma $ è il numero di share che un soggetto deve riuscire
ad ottenere dai agenti affinchè possa
ricostruire il segreto prima del tempo $ \tau $.
Questo numero è ovviamente pari al threshold.
\begin{center}
	$ \gamma = t $
\end{center}


\section{Tecnica di analisi}
Per prima cosa è necessario assegnare ad ogni agente tre attributi con un valore
compreso tra 0 e 1.
\begin{itemize}
	\item \textit{Smarribilità} $ P_s $: la probabilità
	      che l'agente smarrisca il segreto.
	\item \textit{Corruttibilità} $ P_c $: la probabilità che l'agente ceda
	      lo share a lui assegnato ad un altro soggetto.
	\item \textit{Malignità} $ P_m $: la probabilità che l'agente cerchi di
	      corrompere altri agenti per ottenere i loro share.
\end{itemize}

A questo punto si rappresenta l'albero di distribuzione degli share.
Partendo dai nodi foglia si calcolano gli attributi $ P_s $, $ P_c $ e $ P_m $
per i rispettivi nodi padre. Si procede iterativamente in questo modo
fino a quando non si raggiunge il nodo radice. Le $ P_s $, $ P_c $ e $ P_m $
del nodo radice sono le $ P_s $, $ P_c $ e $ P_m $ dell'intera implementazione.

Facciamo un esempio. Sia Carol il client e siano Alice, Bob e Dave i provider.
Siano
$ P_s^A = 0,2 $ \,
$ P_c^A = 0,25 $ \,
$ P_m^A = 0,001 $
gli attributi di Alice;
siano
$ P_s^B = 0,35 $ \,
$ P_c^B = 0,35 $ \,
$ P_m^B = 0,0005 $
gli attributi di Bob;
e siano
$ P_s^C = 0,005 $ \,
$ P_c^C = 0,005 $ \,
$ P_m^C = 0,12 $
gli attributi di Dave.
Ipotizziamo che Carol usi la versione avanzata del protocollo
proposto con numero di share $ n = 3 $ e threshold $ t = 2 $.

Procediamo. Abbiamo che la resistenza ai furti è $ \gamma = 1 $, quindi per calcolare $ P_s $ del nodo
radice dobbiamo trovare la probabilità che vengano smarriti almeno $ \gamma + 1 $ share.
$$ P_s = P_s^A \cdot P_s^B + P_s^A \cdot P_s^C + P_s^B \cdot P_s^C = 0,07275 $$

La resistenza ai furti è $ \theta = 1 $. Analogamente a prima dobbiamo calcolare la probabilità che
almeno $ \theta + 1 $ share vengano sottratti agli agenti.
$$ P_c = P_c^A \cdot P_c^B + P_c^A \cdot P_c^C + P_c^B \cdot P_c^C = 0,0905 $$

Ci resta da trovare $ P_m $. Per ogni agente dobbiamo trovare la probabilità che sia malintenzionato e che riesca
a recuperare gli altri $ n - 1 $ share necessari per ottenere il segreto.
$$ P_m =
	P_m^A \cdot P_c^B + P_m^A \cdot P_c^C +
	P_m^B \cdot P_c^A + P_m^B \cdot P_c^C +
	P_m^C \cdot P_c^A + P_m^C \cdot P_c^B +
	= 0,0724825 $$

\begin{figure}[H]
	\centering
	\includegraphics[width=0.8\linewidth]{images/chap_analisi_robustezza/robustezza-1.pdf}
\end{figure}


% In questo capitolo analizzeremo le principali criticità a cui il protocollo
% proposto è esposto. Per chiarezza di esposizione presenteremo
% i diversi scenari divisi per punti, ma è bene
% sapere che alcune di queste situazioni potrebbero verificarsi anche contemporanemente.

% \section{Comportamenti negligenti dei singoli utenti}
% \subsection{Uno o più agenti perdono il segreto}


% \subsection{L'agente si fa rubare il segreto}

% \subsection{Il client si fa rubare/pubblica il segreto prima di $ \tau $}


% \section{Comportamenti malevoli dei singoli utenti}
% \subsection{Comportamento malevole del client}

% \subsection{Comportamento malevolo di un agente}


% \section{Coalizioni tra agenti}


% \section{Un singolo attore controlla più agenti}
% DLT PERMISSIONED
% Affinché il protocollo funzioni al meglio
% è bene che tutti gli $ N $ agenti
% siano entità ben distinte. Purtroppo non in tutti i contesti
% questa caratteristica è facile da verificare.
% Possono quindi venirsi a creare situazioni nelle quali un unico soggetto controlla
% più agenti. Fissato $ k $ il numero di agenti che controlla,
% individuiamo due casi:
% \paragraph{Un singolo soggetto controlla $ k < \gamma $ agenti}
% In questo caso il soggetto non dispone di un numero sufficiente di share per
% ricostruire il segreto. Bisogna considerare però che parte da una posizione di
% vantaggio, perché deve recuperare solo altri $ \gamma - k $ share.
% \paragraph{Un singolo soggetto controlla $ k \geq \gamma $ agenti}
% In questo caso il soggetto dispone di un numero sufficiente di share per
% ricostruire il segreto. Significa che sin dal momento dell'inizializzazione
% del protocollo viene meno il requisito di
% segretezza del messaggio [vedi \ref{parag:segretezza-tre}].


% \section{Un soggetto corrompe gli agenti}
% Un attaccante che vuole ottenere il messaggio prima del tempo $ \tau $ deve riuscire
% ad ottenere almeno $ \gamma $ share.

% Un tecnica che può usare è quella di "corrompere" un numero $ \gamma $ di agenti.
% Per farlo deve offrire ad ogni agente una cifra maggiore di \textit{prize}.
% L'agente ha interesse nel tenere il proprio share segreto perché sa che se lo fa
% al tempo $ \tau $ può ottenere una ricompensa \textit{prize}.
% Se però un soggetto gli
% offre una cifra maggiore o uguale di \textit{prize} a quel punto l'azione
% più conviente diventa cedere il segreto.

% Come è possibile difendersi? Quello che può fare il client è fissare dei
% \textit{prize} adeguatamente alti rispetto al valore del segreto.
% Un'altra tecnica che può aiutare è quella di
% apportare un'ulteriore penalizzazione nel
% caso in cui l'agente ceda il segreto. Queste penalizzazioni possono essere
% legate al particolare dominio applicativo in cui il protocollo viene utilizzato.
% Un esempio è visibile nell'ultimo capitolo. TODO