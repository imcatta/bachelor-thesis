\chapter{Analisi delle criticità}
\label{chap:analisi-attacchi}
In questo capitolo analizzeremo le principali criticità a cui il protocollo
proposto è esposto. Presenteremo i diversi scenari divisi per punti, ma è bene
sapere che acune di queste situazioni potrebbero verificarsi anche allo stesso
tempo.

\section{Comportamenti negligenti dei singoli utenti}
\subsection{Uno o più provider perdono il segreto}


\subsection{Il provider si fa rubare il segreto}

\subsection{Il client si fa rubare/pubblica il segreto prima di $ \tau $}


\section{Comportamenti malevoli dei singoli utenti}
\subsection{Comportamento malevole del client}

\subsection{Comportamento malevolo di un provider}


\section{Coalizioni tra provider}


\section{Un singolo attore controlla più provider}
DLT PERMISSIONED
Affinché il protocollo funzioni al meglio
è bene che tutti gli $ N $ provider
siano entità ben distinte. Purtroppo non in tutti i contesti
questa caratteristica è facile da verificare.
Possono quindi venirsi a creare situazioni nelle quali un unico soggetto controlla
più provider. Fissato $ k $ il numero di provider che controlla,
individuiamo due casi:
\paragraph{Un singolo soggetto controlla $ k < \gamma $ provider}
In questo caso il soggetto non dispone di un numero sufficiente di share per
ricostruire il segreto. Bisogna considerare però che parte da una posizione di
vantaggio, perché deve recuperare solo altri $ \gamma - k $ share.
\paragraph{Un singolo soggetto controlla $ k \geq \gamma $ provider}
In questo caso il soggetto dispone di un numero sufficiente di share per
ricostruire il segreto. Significa che sin dal momento dell'inizializzazione
del protocollo viene meno il requisito di
segretezza del messaggio [vedi \ref{subsec:segretezza-tre}].


\section{Un soggetto corrompe i provider}
Un attaccante che vuole ottenere il messaggio prima del tempo $ \tau $ deve riuscire
ad ottenere almeno $ \gamma $ share.

Un tecnica che può usare è quella di "corrompere" un numero $ \gamma $ di provider.
Per farlo deve offrire ad ogni provider una cifra maggiore di \textit{prize}.
Il provider ha interesse nel tenere il proprio share segreto perché sa che se lo fa
al tempo $ \tau $ può ottenere una ricompensa \textit{prize}.
Se però un soggetto gli
offre una cifra maggiore o uguale di \textit{prize} a quel punto l'azione
più conviente diventa cedere il segreto.

Come è possibile difendersi? Quello che può fare il client è fissare dei
\textit{prize} adeguatamente alti rispetto al valore del segreto.
Un'altra tecnica che può aiutare è quella di
apportare un'ulteriore penalizzazione nel
caso in cui il provider ceda il segreto. Queste penalizzazioni possono essere
legate al particolare dominio applicativo in cui il protocollo viene utilizzato.
Un esempio è visibile nell'ultimo capitolo.