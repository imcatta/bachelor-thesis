\chapter{Introduzione}

La Blockchain è uno degli argomenti più caldi nell'industria dell'IT. In questo
capitolo cercheremo di fare chiarezza, partendo dalle origini di questa tecnologia
sino ad arrivare allo stato dell'arte odierno.

\section{DLT}
Per capire cosa è una  è necessario
innanzitutto chiarire il concetto di Ledger.

Un \textit{Ledger}, parola che in italiano traduciamo come \textit{Libro Mastro},
è il registro principale in cui sono riunite tutte le transazioni economiche
che compongono un dato sistema contabile. In sostanza si tratta di un documento
nel quale vengono registrate tutte le transazioni economiche, dove viene
indicata la data della transazione, la cifra scambiata e i soggetti coinvolti
nella transazione (chi dà e chi riceve). Le modifiche apportate possono essere solo
di tipo incrementale: non si possono apportare modifiche o cancellare record,
ma solo aggiungere nuove righe al ledger.

Poiché di fatto contiene tutta la storia del sistema contabile, il legder è anche
lo strumento da consultare quando si vuole stabilire quanto denaro possiede un certo
individuo, e se quindi può effettuare un pagamento oppure no per insufficienza
di fondi.

Il Ledger è uno strumento che da secoli viene usato dall'uomo. Le implementazioni
classiche si basano su un soggetto fidato, come ad esempio una banca, che è
l'unico ad essere autorizzato ad apportare modifiche. In questo modo è

Una \textbf{Distribuited Ledger Technology}

\section{Blockchain}
La \textbf{Blockchain} non è altro che una tipologia di DLT. Spesso nell'linguaggio
comune queste due parole vengono usate come sinonimi, ma in questa tesi ci impegneremo
nel mantenere questa distinzione.

Una Blockchain è una catena di transazioni in continua crescita. I transazioni sono
raggruppati in blocchi, e questi blocchi sono collegati tra di loro con tecniche
crittografiche.

Ogni blocco è composto da:
\begin{itemize}
	\item L'hash crittografico del blocco precedente
	\item Un timestamp, ossia la data e l'ora dell'aggiunta del blocco alla catena
	\item Un insieme di transazioni
\end{itemize}
La Blockchain è stata inventata da Satoshi Nagamoto nel 2008 per essere utilizzata
come ledger pubblico per la gestione della criptovaluta Bitcoin. Con la blockchain,
per la prima volta si è potuto risolvere il problema del \textit{double-spending}
senza ricorrere ad una autorità fidata.

\section{Smart Contracts}
Uno \textbf{Smart Contract} è un protocollo computerizzato che ha lo scopo di
facilitare, verificare o obbligare la negoziazione o l'esecuzione di un contratto.
L'aspetto sicuramente più interessante è che permettono di garantire l'esecuzione
di determinate azioni, come un pagamento, senza l'intervento di un soggetto terzo.
Gli Smart Contracts furono pensati per la prima volta da Nick Szabo nel 1994, per poi
essere formalizzati dal lui stesso nel 1997 (aggiungere riferimento al paper).

\section{Stellar}

\textbf{Stellar} è un protocollo decentralizzato ed
open source nato con lo scopo di poter
trasferire denaro attraverso diversi
paesi con costi di transazione irrisori. Il nome della criptovaluta scambiata
è Lumen (\textbf{XLM}).

È stato creato nel 2014 da \textbf{Jed McCaleb} e \textbf{Joyce Kim}. 
Si tratta di un progetto fortemente ispirato da \textit{Ripple} (di cui Jed McCaleb 
è co-fondatore). A differenza di quest'ultimo, Stellar si pone come obietivo
il facilitare lo scambio di denaro tra privati invece che tra banche.

Ad oggi l'invio di denaro ad una persona che vive in una regione geografica 
diversa deve passare attraverso un elevato numero di intermediari. Ciò comporta
una lentezza e un elevato costo di transazione.
...


\subsection{Gli Smart Contracts in Stellar}
In Stellar gli Smart Contracts di Szabo sono implementati
attraverso gli Stellar Smart Contracts.

Uno \textbf{Stellar Smart Contract} (SSC) è espresso come la composizione
di transazioni connesse fra loro ed eseguite secondo certi vincoli.
I vincoli che si possono utilizzare nella realizzazione di SSCs sono:

\begin{itemize}
	\item \textit{Multisignature (multifirma)} - Quali chiavi sono necessarie
	      per autorizzare
	      una certa transazione? Quali soggetti devono concordare in una certa
	      circorstanza affinche si possano eseguire i passi?
\end{itemize}
La Multisignature è il concetto di richiedere firme di soggetti diversi per
effettuare transazioni provenienti da un certo account.
Attraverso pesi e soglie di firma,
viene creata la rappresentazione del potere nelle firme.

\begin{itemize}
	\item \textit{Batching/Atomicity (batching/atomicità)} -
	      Quali operazioni devono avvenire
	      o tutte insieme o fallire?
	      Cosa deve accadere per forzare il successo o il fallimento?
\end{itemize}
Il Batching è il concetto dell'includere più operazioni in un'unica transazione.
L'atomicità è la garanzia che, data una serie di operazioni raggruppate in
un'unica transazione,
al momento dell'invio sul network se anche una sola operazione fallisce,
tutte le operazioni nella transazione falliscono.

\begin{itemize}
	\item \textit{Sequence (sequenza)} -
	      In che ordine deve essere elaborata una serie di transazioni?
	      Quali sono le limitazioni e le dipendenze?
\end{itemize}
Il concetto di sequenza è rappresentato sullo Stellar network attraverso
il numero di sequenza. Utilizzando i numeri di sequenza
nella manipolazione delle transazioni è possibile garantire che
transazioni specifiche non possano essere eseguite
se viene inoltrata una transazione alternativa.

\begin{itemize}
	\item \textit{Time Bounds (limiti temporali)} -
	      Quando è possibile elaborare una transazione?
\end{itemize}
I limiti di tempo sono vincoli sul periodo di tempo durante il quale
una transazione è valida. L'utilizzo dei limiti di tempo
consente di rappresentare i periodi di tempo in un SSC.
