\chapter{Timed-Release Encryption}

\section{Cosa è la Timed-Release Encryption}
La \textbf{Timed-Release Encryption} (TRE)
è un metodo per cifrare un messaggio in modo che possa
essere decifrato solo dopo che una certa deadline è stata raggiunta.
Un avversario, anche se ha a disposizione una grande quantità di potenza di calcolo,
non deve essere in grado di ottenere il messaggio prima della deadline.
Inoltre, dopo la deadline i destinatari devono poter leggere il messaggio,
indipendentemente dalla potenza di calcolo che hanno a disposizione e
senza dover interagire con il mittente,
con altri destinatari o con terze parti fidate.

Questo tipo di crittografia è molto difficile da ottenere con un modelli
di computazione standard (come la macchina di Turing), perché questi non mettono
a disposizione un "orologio del mondo reale" affidabile.
Nel capitolo \ref{chap:protocollo} proponiamo una soluzione che si appoggia sui DLT,
i quali mettono a disposizione
le buone proprietà di cui necessita la Timed-Release Encryption.

Il problema dell'\textit{invio di un messaggio nel futuro} è stato proposto per la prima volta
da May \cite{May:time-released-crypto} nel 1993 e
successivamente studiato da Rivest, Shamir and Wagner \cite{Rivest96time-lockpuzzles}.

La Timed-Release Encryption ha diverse applicazioni nel mondo reale. Queste verranno
discusse nel dettaglio nel capitolo \ref{chap:possibili-usi}.

\section{Requisiti}
In sintesi, un buon sistema di TRE deve soddisfare i seguenti requisiti:

\paragraph{Certezza di pubblicazione}
\label{parag:certezza-pubblicazione}
Un sistema di TRE deve garantire con un buon
grado di confidenza che al tempo $ \tau $ il messaggio venga reso noto.

\paragraph{Segretezza del messaggio}
\label{parag:segretezza-tre}
Un sistema di TRE deve garantire con un buon
grado di confidenza che il messaggio rimanga segreto sino al tempo $ \tau $.


\section{Implementare la Timed-Release Encryption}
Le tecniche di implementazione della Timed-Release Encryption possono essere classificate
in due categorie.

\subsection{Time-Locked Puzzle}
I Time-Locked puzzle sono stati introdotti per la prima volta
da Merkle \cite{Merkle:1978:SCO:359460.359473} e approfonditi
da Bellare et. al. \cite{Bellare:1996:EKE:888619}
e da Rivest et. al. \cite{Rivest96time-lockpuzzles}.
L'idea che sta alla base è quella di trasformare il messaggio ad un certo istante di tempo $ t_0 $
in modo tale che ogni macchina
(seriale o parallela) debba lavorare per un certo quantitativo di tempo $ \Delta $ per risolvere
il problema computazionale sottostante (puzzle). La somma tra $ t_0 $ e $ \Delta $ è la deadline.

Sebbene rappresenti un approccio elegante nell'ambito della teoria della complessità computazionale,
l'uso di \textit{Time-Locked Puzzle} è difficilmente sostenibile nella pratica,
poiché è poco flessibile e richiede l'uso di
grandi quantità di risorse di calcolo. Inoltre è necessario conoscere con
precisione la potenza di calcolo a disposizione del risolutore per poter
dimensionare la difficoltà del puzzle. Generalmente è possibile ottenere
una buona stima di questo valore se ci si basa sull'hardware contemporaneo. È impossibile
invece trovare una stima \textit{future-proof}, poiché non sappiamo quali nuovi strumenti di
calcolo verranno creati nel futuro (es. computer quantistici).
Ad ogni modo, anche se si avesse una stima affidabile a dispozione,
sarebbe possibile garantire solo un limite inferiore
sull'istante di pubblicazione del messaggio, dato che la macchina risolutrice potrebbe per vari motivi
non iniziare immediatamente a lavorare all'istante $ t_0 $,
potrebbe effettuare delle pause durante l'elaborazione
o peggio ancora potrebbe non completare la risoluzione del problema.


\subsection{Trusted agents (agenti fidati)}
\label{subsec:trusted-agents}
Questo approccio si basa su uno o più \textit{agenti fidati} che rilasciano
delle informazioni una volta che si è raggiunta la deadline. Questo può avvenire in
maniera interattiva o non interattiva. Alcuni riferimenti
\cite{time-capsule-signature}, \cite{10.1007/11602897_25}, \cite{10.1007/3-540-48910-X_6},
\cite{10.1007/11889663_17}, \cite{10.1007/978-3-642-15317-4_1}, \cite{10.1145/1330332.1330336}.

I vantaggi di questa strategia sono che non è richiesto che venga effettuata una computazione non-stop,
è possibile di rispettare con precisione la deadline e non è richiesta una stima precisa
della capacità di calcolo dei soggetti coinvolti. Tutto ciò ha un prezzo: gli agenti devono
essere affidabili e devono essere attivi allo scoccare della deadline.

