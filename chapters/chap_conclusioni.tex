\chapter{Conclusioni}
In questa tesi abbiamo proposto un protocollo per la Timed-Release Encryption. A differenza delle soluzioni precedenti \cite{time-capsule-signature, 10.1007/11602897_25, 10.1007/3-540-48910-X_6,
	10.1007/11889663_17, 10.1007/978-3-642-15317-4_1, 10.1145/1330332.1330336, chalkias2007improved},
il protocollo proposto è profondamente decentralizzato. Il messaggio è distribuito tra $ N $ agenti orchestrati da smart contract eseguiti su blockchain. Nella gran parte delle situazioni il protocollo proposto è in grado di fornire un buon grado di robustezza, ossia è in grado di resistere ad un certo numero di comportamenti scorretti da parte degli agenti senza compromettere la \textit{certezza di pubblicazione} e la \textit{segretezza del messaggio}.

I possibili sviluppi futuri sono diversi. È possibile valutare l'introduzione di uno strumento che utilizza la \textit{zero knowledge proof} per permettere ai destinatari del messaggio di verificare che gli agenti non abbiano smarrito gli share senza dover attendere la deadline. Si può studiare come aggiungere ulteriori disincentivi per dissuadere gli agenti da comportamento scorretti, ad esempio introducendo delle sanzioni attraverso contratti giuridici, appoggiandosi a ledger \textit{permissioned} e meccanismi \textit{KYC}\footnote{know your customer} per validare gli agenti. Infine, si può valutare come gestire messaggi dalle elevate dimensioni che non possono essere economicamente memorizzati nella blockchain utilizzando prodotti commerciali per lo storage in cloud come Amazon S3 \cite{amazon-s3} e Google Cloud Storage \cite{gcs} oppure utilizzando storage distribuiti come IPFS \cite{ipfs} e Ethereum Swarm \cite{ethereum-swarm}.
